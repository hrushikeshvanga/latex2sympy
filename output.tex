\documentclass{article}
\usepackage[utf8]{inputenc}
\usepackage{amsmath, amsfonts, mathtools}
\usepackage[bottom=1in, top=0.5in]{geometry}
\usepackage{setspace}
\newcommand{\pmat}[4]{\begin{pmatrix} #1 & #2 \\ #3 & #4\end{pmatrix}}
\newcommand{\R}[0]{\mathbb{R}}

\title{MATH 225 Assignment 7}
\author{Curtis Kan}
\date{2022-03-17}

\setstretch{1.25}
\begin{document}

\maketitle

\section*{Question 1}
\subsection*{1.}
A. T(f(X)) = a times X squared + b times X + c = a times opening bracket 2 X + 1 closing bracket to the power 2 + b times opening bracket 2 X + 1 closing bracket + c = a times opening bracket 4 times X squared + 4 X + 1 closing bracket + 2 times b times X + b + c \\ = opening bracket 4 a closing bracket times X squared + opening bracket 4 a + 2 b closing bracket times X + opening bracket a + b + c closing bracket
\\a times X squared + b times X + c = opening bracket 4 a closing bracket times X squared + opening bracket 4 a + 2 b closing bracket times X + opening bracket a + b + c closing bracket
\\Let's use the standard basis for **Error** and make a matrix for T:
\\T of 1 = 1 = 1 times 1 + 0 times X + 0 times opening bracket X squared closing bracket 
\\T of X = 2 X + 1 = 1 times 1 + 2 X + 0 times opening bracket X squared closing bracket
\\T of X^2 = 4 times X squared + 4 X + 1 = 1 times 1 + 4 X + 4 times opening bracket X squared closing bracket
\\[10px]
MatrixSymbol(Str('T'), Integer(1), Integer(13))
\\[Integer(1), Integer(2), Integer(4)] geometric multiplicity = 3, algebraic multiplicity  = 3
\\B. **Error**
\\C. Diagonalizable since g times e times o = a times l times g
\\D. **Error**
\\MatrixSymbol(Str('T'), Integer(1), Integer(13))
\subsection*{2.}
A. T(f(X)) = a times X squared + b times X + c
\\f(X) = a times opening bracket X + 1 closing bracket to the power 2 + b times opening bracket X + 1 closing bracket + c = a times opening bracket X squared + 2 X + 1 closing bracket + b times X + b + c = a times X squared + opening bracket 2 a + b closing bracket times X + opening bracket a + b + c closing bracket
\\a times X squared + b times X + c = a times X squared + opening bracket 2 a + b closing bracket times X + opening bracket a + b + c closing bracket
\\Let's use the standard basis for **Error** and make a matrix for T:
\\T of 1 = 1 = 1 times 1 + 0 times X + 0 times opening bracket X squared closing bracket 
\\T of X = X + 1 = 1 times 1 + 1 times X + 0 times opening bracket X squared closing bracket
\\T of X^2 = X squared + 2 X + 1 = 1 times 1 + 2 X + 1 times opening bracket X squared closing bracket
\\[10px]
MatrixSymbol(Str('T'), Integer(1), Integer(13))
\\Integer(1) geometric multiplicity = 1, algebraic multiplicity  = 3
\\B. **Error**
\\C. Not diagonalizable since **Error**
\\D. Not diagonalizable so can't find.
\subsection*{3.}
A.T = Matrix([[1, 1, 0, 0, 2, 0, 0, 0, 0, 10, 0, -1, 0]])
\\B. [Integer(2), Integer(1)] geometric multiplicity = 4 algebraic multiplicity = 2
\\C. Not diagonalizable since there are some complex eigenvalues but aren't included.
\\D. Not diagonalizable so can't find.

\subsection*{4.}
A.MutableDenseMatrix([[Integer(1), Integer(-1), Integer(0), Integer(0), Integer(2), Integer(0), Integer(0), Integer(0), Integer(0), Integer(-10), Integer(0), Integer(1), Integer(0)]])
\\B.['-1 times i', Symbol('i')]  geometric multiplicity = 2 algebraic multiplicity = 2
\\C. Diagonalizable since g times e times o = a times l times g
\\D. **Error**
\pagebreak
\section*{Question 2}
\subsection*{1.}
Suppose there is an arbitrary vector **Error** such that **Error** and **Error** 
\\(v is contained in the eigenspace of Symbol('lambda_1') and Symbol('lambda_2')), then we know that \\ T of v = lambda_1 times v and T of v = lambda_2 times v by Definition 14.5. By combining these equations through T of v, we get:
\\lambda_1 times v = lambda_2 times v. 
\\Rearranging, we get:
\\lambda_2 times v minus lambda_1 times v = 0. 
\\We know that the eigenvalues Symbol('lambda_1') and Symbol('lambda_2') are 2 numbers that are distinct, thus **Error**. So we must have Integer(0). Hence, the only vector contained in Symbol('T_{\\lambda_1}') and Symbol('T_{\\lambda_2}') is the zero vector.
\subsection*{2.}
We want to show the set **Error** is linearly independent. Let's show the only solution: \\ **Error** is when **Error**.
\\We know **Error** = lambda_1 times v_r since **Error** is a subset of Symbol('T_{\\lambda_1}').
\\We know **Error** = lambda_2 times w_s since **Error** is a subset of Symbol('T_{\\lambda_2}').
\\[5px]Equation 1: take 0 = T(0) = T(**Error**) \\ = **Error** = 0. 
\\[5px]Equation 2: take the original equation and multiply by Symbol('lambda_1'): **Error** = 0.
\\Subtract Equation 1 - Equation 2 (all v terms cancel): (lambda_2 minus lambda_1)d_1 times w_1 + ... + (lambda_2 minus lambda_1)d_s times w_s = 0.
\\**Error** = 0.
\\We know from part 1 that **Error**, so **Error**. **Error** is linearly independent, thus by definition **Error** = 0. 
\\[5px]We can use the same method with **Error** by modifying Equation 2 to multiply by Symbol('lambda_2'):
\\**Error** = 0.
\\Subtract Equation 2 - Equation 1 (all w terms cancel): (lambda_2 minus lambda_1)c_1 times v_1 + ... + (lambda_2 minus lambda_1)c_r times v_r = 0.
\\**Error** = 0.
\\We know from part 1 that **Error**, so **Error**. **Error** is linearly independent, thus by definition **Error** = 0. 
\\We proved that **Error**. Thus  **Error** is a linearly independent subset of V.
\pagebreak
\section*{Question 3}
By Definition 14.5, we know the geometric multiplicity of Symbol('lambda') is the dimension of eigenspace Symbol('T_{\\lambda}'). If the eigenspace agrees with range(T), that must mean geometric multiplicity Symbol('lambda') = dim(T), which by definition means there can be no other eigenvalues. Hence, Symbol('lambda') is the only eigenvalue. By Theorem 15.3, the sum of geometric multiplities is just Symbol('lambda'), which is dim(T), so therefore T is diagonalizable.
\\Proof by intimidation.


\end{document}